\documentclass[11pt]{article}

\usepackage{listings}
\usepackage[usenames,dvipsnames]{xcolor}
\usepackage[margin=1in]{geometry}

\title{Proposal: VDIFParse}
\author{Mars Buttfield-Addison}
\date{\today}

\definecolor{bluekeywords}{rgb}{0.13,0.13,1}
\definecolor{greencomments}{rgb}{0,0.5,0}
\definecolor{redstrings}{rgb}{0.9,0,0}
\lstset{
	language=C,
	basicstyle=\ttfamily,
	tabsize=4,
	showspaces=false,
	showtabs=false,
	breaklines=true,
	showstringspaces=false,
	breakatwhitespace=true,
	escapeinside={(*@}{@*)},
	keywordstyle=\color{bluekeywords},
	commentstyle=\color{greencomments},
	stringstyle=\color{redstrings},
}

\begin{document}
\maketitle

\section{Purpose}

Currently, the need to ingest and manipulate telescope data in the modern standard VDIF (or variant CODIF) formats is met by DiFX with its \texttt{mark5access} and \texttt{vdifio} libraries. However, the need for \texttt{mark5access} to also support several legacy formats has prevented it from using practices best suited to the VDIF format, at a significant cost to performance. The limited functionality of the \texttt{vdifio} library could not be incorporated into \texttt{mark5access} without suffering the inherent slowdown to multithreaded VDIF processing.

This presents an opportunity to create a new library that combines the functionality of the two previous libraries, designed especially for performance with the VDIF format. Coupled with a modern, easy-to-use API, such a library could also fill a gap that would allow quick creation of small VDIF processing programs. Herein proposes such a solution, called \texttt{VDIFParse}.

\section{Assumed Use Cases}

\begin{itemize}
\item Someone wants to reorganize/filter data into files that will be fast to decode or analyse later.
\item Someone wants to know the properties of the data they have.
\item Someone wants to 	
\end{itemize}


\section{Required Functionality}

\subsection{File Management}

To make it easier for later analysis, some file management utilities should be included that will split or ``clean up'' files. This will not perform any analysis or decoding of the file contents beyond that required to move frames around.

\begin{itemize}
	\item \textbf{Split file}: take in a file path (e.g., \texttt{~/example.vdif}) and output new files for each thread or thread group (e.g., \texttt{~/example\_thread0.vdif}). This can be used to turn compound data streams into multiple simple data streams, or save later work in loading all threads from a file just to decode one.
	\item \textbf{Clean file}: take in a file path and output a new file that has been skew corrected--meaning frames have been sorted in order of seconds from epoch and then thread id--and inserted empty invalid frames for any time gaps.
	\item \textbf{Split and clean file}: perform split file and also clean each new file as it is being written to.
\end{itemize}

\subsection{Data Summary}

\subsection{Data Dispatch}

\subsection{Data Decode}

\section{Technologies}

Library is to be written in native C, in conformance with \textit{at least} GNU90 ad C99. Dependencies will be kept to the C Standard Library. The library should target both OSX and Debian-based Linux.

\section{API Design}

Usage will centre around a struct type \lstinline{DataStream}, which can be initialised in one of two modes: \lstinline{FileMode} or \lstinline{StreamMode}.

In \lstinline{FileMode}, the expected interaction is that a user will initialise an \lstinline{InputStream} using the \lstinline{open_file()} function and passing a filepath to a valid VDIF or CODIF file.

\begin{lstlisting}
#include <stdio.h>
#include "vdifparse.h"

int main() { 
	int res; // holds result status code for each operation
	
	// open new stream
	DataStream ds;
	open_file("file.vdif", &ds);
	
	// set format designator (saves summarising the file in advance)
	set_format_designator(&ds, "VDIF-1024-16-2-4");
	
	// select specific threads to include in output
	int selected_threads[2] = { 0, 1 };
	set_output_threads(&ds, selected_threads);
	
	// decode some samples
	float** output;
	DecodeMonitor statistics;
	int num_samples = 20000;
	decode_samples(&ds, num_samples, &output, &statistics);
	
	// free the memory allocated for the stream
	close(&ds);
	
	// (do something with output here)
} 
\end{lstlisting} 

In \lstinline{StreamMode}, the expected interaction is that a user will initialise an (initially empty) \lstinline{InputStream} using the \lstinline{open_stream()} function and then push raw data into it by monitoring a specific port or piping from another process.

\begin{lstlisting} 
#include "vdifparse.h"

int main() { 
	// open new stream (most values remain unset)
	DataStream ds = open_stream();
	
	
	close(in);
} 
\end{lstlisting} 

\section{Style and Conventions}

\begin{itemize}
	\item Type names are in PascalCase (e.g., \lstinline{DataStream}).
	\item Function and variable names are in snake\_case (e.g., \lstinline{sample_rate}).
\end{itemize}

\section{Collected Prompts for Feedback}

\begin{enumerate}
	\item Nowadays, the most popular architectures (x86-64, IA-32, etc.) all use little-endianness. Can you think of a device or prospective used that would still require support for big-endianness?
	\item The VDIF and CODIF specs allow for anywhere from 1-bit to 32-bit samples, but existing software only shows support for 1, 2, 4 or 8-bit decode. Should support be added for the full range of sample options?
\end{enumerate}

	
\end{document}
