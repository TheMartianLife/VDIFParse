\documentclass[11pt]{article}

\usepackage{listings}
\usepackage[usenames,dvipsnames]{xcolor}

\title{Proposal: VDIFParse}
\author{Mars Buttfield-Addison}
\date{\today}

\definecolor{bluekeywords}{rgb}{0.13,0.13,1}
\definecolor{greencomments}{rgb}{0,0.5,0}
\definecolor{redstrings}{rgb}{0.9,0,0}
\lstset{
	language=C,
	basicstyle=\ttfamily,
	tabsize=4,
	showspaces=false,
	showtabs=false,
	breaklines=true,
	showstringspaces=false,
	breakatwhitespace=true,
	escapeinside={(*@}{@*)},
	keywordstyle=\color{bluekeywords},
	commentstyle=\color{greencomments},
	stringstyle=\color{redstrings},
}

\begin{document}
\maketitle

\section{Purpose}

Currently, the need to ingest and manipulate telescope data in the modern standard VDIF (or variant CODIF) formats is met by DiFX with its \texttt{mark5access} and \texttt{vdifio} libraries. However, the need for \texttt{mark5access} to also support several legacy formats has prevented it from using practices best suited to the VDIF format, at a significant cost to performance. The limited functionality of the \texttt{vdifio} library could not be incorporated into \texttt{mark5access} without suffering the inherent slowdown to multithreaded VDIF processing.

This presents an opportunity to create a new library that combines the functionality of the two previous libraries, designed especially for performance with the VDIF format. Coupled with a modern, easy-to-use API, such a library could also fill a gap that would allow quick creation of small VDIF processing programs. Herein proposes such a solution, called \texttt{VDIFParse}.

\section{Use Cases}

\section{Technologies}

Library is to be written in native C, in conformance with \textit{at least} GNU90 ad C99. Dependencies will be kept to the C Standard Library. The library should target both OSX and Debian-based Linux.

\section{API Design}

Usage will centre around a struct type \lstinline{InputStream}, which can be in one of two modes: \lstinline{FileMode} or \lstinline{StreamMode}.

In \lstinline{FileMode}, the expected interaction is that a user will initialise an \lstinline{InputStream} using the \lstinline{open_file()} function and passing a filepath to a valid VDIF or CODIF file.

\begin{lstlisting} 
#include "vdifparse.h"

int main() { 
	// open new stream (also parses first header)
	struct InputStream* in = open_file("file.vdif");
	
	
	close(in);
} 
\end{lstlisting} 

In \lstinline{StreamMode}, the expected interaction is that a user will initialise an (initially empty) \lstinline{InputStream} using the \lstinline{open_stream()} function and then push raw data into it by monitoring a specific port or piping from another process.

\begin{lstlisting} 
#include "vdifparse.h"

int main() { 
	// open new stream (most values remain unset)
	struct InputStream* in = open_stream();
	
	
	close(in);
} 
\end{lstlisting} 

\section{Style and Conventions}

\begin{itemize}
	\item Type names are in PascalCase (e.g., \lstinline{InputStream}).
	\item Function and variable names are in snake\_case (e.g., \lstinline{sample_rate}).
\end{itemize}

\section{Collected Prompts for Feedback}

\begin{enumerate}
	\item Nowadays, the most popular architectures (x86-64, IA-32, etc.) all use little-endianness. Can you think of a device or prospective used that would still require support for big-endianness?
	\item In the current form, it is technically possible to use one \lstinline{InputStream} in both \lstinline{FileMode} and \lstinline{StreamMode} at different times (e.g., open a file and then later stream more data in). Can you think of any conceptual reasons to allow or disallow this?
	\item In the style of \texttt{mark5access}, it may be advisable to add a domain marker such as a \lstinline{vp_} prefix to all types or functions, to prevents overloading when imported into other projects. Can you think of any reason not to do this?
\end{enumerate}

	
\end{document}
